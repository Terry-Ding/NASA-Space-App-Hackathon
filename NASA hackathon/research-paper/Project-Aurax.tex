\documentclass[12pt, a4paper]{article}

% --- Packages ---
\usepackage[utf8]{inputenc}
\usepackage{graphicx}
\usepackage{geometry}
\usepackage{hyperref}
\usepackage{amsmath}
\usepackage{booktabs}
\usepackage{array} % for better column formatting
\usepackage{enumitem}

% --- Page setup ---
\geometry{margin=1in}

\title{\textbf{Mars Habitat Resource Recycling \& Manufacturing System}}
\author{%
    \textbf{Team Name:} Project Aura\textquotesingle x \\[0.3em]
    \textbf{Members:} 
}
\date{\today}

\begin{document}

\maketitle

\section{Problem Definition}

\subsection{Need Statement}
During a three-year mission, an eight-person crew would accumulate over 12,000 kg of inorganic waste, or trash, which is a huge problem. If the issue is not addressed as soon as possible, we humans might cause some permanent damage to the environment on Mars.

\subsection{Goal}
Find an efficient and reliable way to recycle and reuse the waste on Mars.

\subsection{Objectives}

\begin{table}[h!]
\centering
\renewcommand{\arraystretch}{1.3} % line spacing
\setlength{\tabcolsep}{8pt} % column spacing
\begin{tabular}{p{0.25\textwidth} p{0.45\textwidth} p{0.2\textwidth}}
\toprule
\textbf{Objective} & \textbf{Basis for Measurement} & \textbf{Units} \\ 
\midrule
Must be reliable & The endurance of the solution (how many years it can last) & Years \\[0.3em]
Must not be too expensive & The cost to implement the solution & CAD \\[0.3em]
Must minimize the emission & The amount of chemical the solution might emit & -- \\[0.3em]
Must be buildable by no more than eight people & Number of people required for the workload & People \\
\bottomrule
\end{tabular}
\caption{Objectives and Measurement Basis}
\end{table}

\newpage 
\subsection{Constraints}
\begin{itemize}
    \item Solutions must be completed within one week.
    \item Solutions must not burn anything without proper collection on Mars.
    \item Solutions must not cost too much money.
    \item Solutions must not be too complicated to implement.
\end{itemize}

\section{Our Solution: Mars Habitat Recycling-to-Manufacturing System (MHRMS)}

\subsection{System Overview}
\textbf{Purpose:} Convert waste materials from Mars habitat operations into usable 3D printing filament and manufacturing feedstock, with emphasis on CO2 extraction byproducts.

\textbf{Design Philosophy:} Modular, relocatable, powered by habitat grid, zero-waste discharge

\textbf{Total System Mass:} $\sim$720 kg (disassembled into $<$100 kg components)

\subsection{Mars Environment Context}
\begin{enumerate}
    \item Mars has a thin atmosphere made up mostly of carbon dioxide, nitrogen, and argon gases, with temperatures ranging from 70°F (20°C) to -225°F (-153°C)
    \item Mars' sparse atmosphere doesn't offer much protection from impacts by meteorites and the thin atmosphere causes heat from the Sun to easily escape
    \item Dust storms can cover much of the planet, sometimes taking months for all dust to settle
    \item Mars completes one rotation every 24.6 hours (a "sol"), with a year lasting 687 Earth days
    \item These conditions directly impact recycling system design: thermal management is critical, dust mitigation essential
\end{enumerate}

\subsection{Critical Mars-Specific Design Considerations}

\subsubsection{Waste Behavior in Mars Environment}
\begin{itemize}
    \item \textbf{Vacuum Exposure Challenges:} Based on NASA Glenn Research Center analysis of waste behavior when exposed to vacuum
    \item \textbf{Water Sublimation:} When waste is exposed to Mars' near-vacuum conditions (0.6\% Earth pressure), approximately 20\% of water content will sublimate before the waste completely freezes. This process takes:
    \begin{itemize}
        \item Small items (3mm droplets): $\sim$7 seconds to freeze
        \item Standard waste "football" (20cm compressed bundle): $\sim$3.4 hours to freeze
    \end{itemize}
    \item \textbf{Implications for Recycling System:}
    \begin{enumerate}
        \item Pre-processing must handle both frozen and partially sublimated materials
        \item Vapor capture systems required to prevent water loss during initial processing
        \item Grinding chambers must be sealed to capture sublimating volatiles
        \item Temperature control critical: waste arriving at facility may be frozen solid (-153°C from overnight exposure)
    \end{enumerate}
\end{itemize}

\subsubsection{Thermal Management on Mars}
\begin{itemize}
    \item \textbf{Challenge:} Mars temperature swings from 70°F (20°C) to -225°F (-153°C)
    \item \textbf{Solutions:}
    \begin{itemize}
        \item Insulated processing chambers: Maintain optimal processing temperatures (150-1400°C depending on module)
        \item Pre-heating systems: Thaw frozen waste before processing
        \item Waste heat recovery: Capture heat from processing for habitat use
        \item Phase-change thermal buffers: Smooth out temperature fluctuations
    \end{itemize}
\end{itemize}

\subsubsection{Mars Sol Adaptation}
\begin{itemize}
    \item \textbf{Mars Day:} 24.6 hours (24h 39m) vs Earth's 24 hours
    \item \textbf{Operational Impact:}
    \begin{itemize}
        \item Processing schedules calibrated to Mars sols
        \item Automation advantage: rovers and processing continue during crew sleep cycles
        \item Benefit: Extra 39 minutes per sol allows for extended processing cycles
    \end{itemize}
\end{itemize}

\noindent\textit{Source:} \href{https://science.nasa.gov/mars/facts/}{NASA Mars Facts}.

\section{Input Categories}

\begin{enumerate}
    \item \textbf{Category A -- Carbon Materials:}
    \begin{itemize}
        \item Surplus carbon from CO2 extraction (primary feedstock)
        \item Activated carbon filters (spent)
        \item Carbon-containing composites
    \end{itemize}

    \item \textbf{Category B -- Polymers/Plastics:}
    \begin{itemize}
        \item Nitrile gloves
        \item Resealable bags (PE/PP)
        \item EVA waste from spacesuits and cargo transfer bags (Nomex, nylon, polyester)
        \item Packaging materials (air cushions, bubble wrap, anti-static bags)
        \item Tape, labels, plastic clips
        \item Foam packaging (Zotek F30, Plastazote foam)
    \end{itemize}

    \item \textbf{Category C -- Fabrics:}
    \begin{itemize}
        \item Clothing (cotton, polyester, nylon)
        \item Washcloths and wipes (cellulose, cotton)
        \item Disinfectant wipes 
    \end{itemize}

    \item \textbf{Category D -- Food Packaging:}
    \begin{itemize}
        \item Overwrap materials
        \item Rehydratable pouches (polyester, polyethylene, aluminum, nylon)
        \item Drink pouches
        \item Thermal pouches
    \end{itemize}

    \item \textbf{Category E -- Metals:}
    \begin{itemize}
        \item Filter mesh (stainless steel, aluminum)
        \item Broken tools and fasteners
        \item Structural components from temporary structures
        \item Electronics casings
        \item Wire scraps
        \item Aluminum structures
    \end{itemize}

    \item \textbf{Category F -- Composites:}
    \begin{itemize}
        \item Polymer matrix composites with carbon fiber
        \item Multi-layer materials
        \item Fabric-backed materials
        \item Adhesive-bonded assemblies
    \end{itemize}
\end{enumerate}

\subsection{NASA Data Reference}
Based on ISS and shuttle waste studies, daily waste generation averages 1.45 kg per crew member, including 0.160 kg clothing, 0.352 kg food/packaging, 0.449 kg human waste, and other consumables. For a 4-person crew on a 180-day mission, this totals approximately 1,046 kg of recyclable waste.

\subsection{Sorting Protocol}
\begin{itemize}
    \item Manual inspection - crew separates by visual category
    \item Magnetic separation - ferrous vs non-ferrous metals
    \item Density sorting - polymer type identification
    \item Quality control - contamination check, damage assessment
    \item Storage: Separate sealed containers for each category in airlock staging area
\end{itemize}

\section{Alternative Waste Processing Options}

\begin{enumerate}
    \item \textbf{Option A -- Direct Venting to Space (NOT RECOMMENDED)}
    
    NASA Glenn Research Center evaluated simply disposing waste via airlock to space. Key findings:
    
    \textbf{Technical Challenges:}
    \begin{enumerate}
        \item Time-intensive: Using existing airlock technology (3-hour depressurization cycle), disposing of waste from a 180-day, 4-person mission would require 1,250-2,500 hours of crew time
        \item Hazard: 20\% of water content sublimates and can re-condense on spacecraft surfaces or contaminate equipment
        \item Trajectory risk: Low-velocity disposal at L2 libration points risks waste re-impacting spacecraft
    \end{enumerate}
    \textbf{Conclusion:} Direct disposal to space is impractical and hazardous for Mars missions

    \item \textbf{Option B: Partial Processing to Mixed Gases}
    
    NASA's "Trash-to-Gas" (TtG) Approach: Process waste in primary reactor only, producing mixed gases:
    \begin{itemize}
        \item 10\% H$_2$, 22\% CO, 68\% CO$_2$
        \item Can be used in resistojet thrusters (134.3 sec Isp)
        \item Requires 0.28 kg/crew-day makeup water
    \end{itemize}
    
    \textbf{Benefits:}
    \begin{itemize}
        \item Simpler system (fewer processing steps)
        \item Can provide station-keeping propulsion
        \item Lower energy requirements
    \end{itemize}
    
    \textbf{Limitations:}
    \begin{itemize}
        \item Lower-quality propellant (vs. pure O$_2$/CH$_4$)
        \item Still requires makeup water
        \item Limited manufacturing applications
    \end{itemize}

    \item \textbf{Option C: Full Processing to High-Quality Propellants (RECOMMENDED)}
    
    NASA's "Trash-to-Supply Gas" (TtSG) - Our Baseline: Complete processing through secondary reactors:
    \begin{itemize}
        \item 59\% O$_2$, 41\% CH$_4$ (optimal rocket propellant)
        \item Can achieve 250+ sec Isp in rocket engines
        \item Enables both propulsion AND manufacturing uses
        \item Requires 0.15 kg/crew-day makeup water (less than TtG)
    \end{itemize}
\end{enumerate}

\subsection{Our Enhanced System vs NASA's Approach}

\textbf{NASA TtSG (gases only):}
\begin{itemize}
    \item Produces O$_2$ and CH$_4$ for propulsion
    \item Vents excess gases or stores in tanks
    \item Limited to propellant applications
\end{itemize}

\textbf{Our Integrated System (gases + solids + ISRU):}
\begin{itemize}
    \item Processes waste into filament AND propellant gases
    \item Combines with regolith to create metals, glass, ceramics
    \item 3D prints replacement parts, tools, habitat components
    \item Enables true self-sufficiency vs. just propellant production
    \item Creates complete manufacturing economy
\end{itemize}

\textbf{Result:} Our system achieves everything NASA's TtSG does PLUS enables permanent settlement through manufacturing capability.

\section{System Modules}

\subsection{Module 1: Carbon Processing Station}

\textbf{Equipment Components:}
\begin{itemize}
    \item Enclosed ball mill grinder (50 kg)
    \item Compression mold press (75 kg)
    \item Electric heating elements (25 kg)
    \item Polymer binder mixing chamber (25 kg)
    \item Extrusion die set (15 kg)
    \item Control electronics (10 kg)
    \item \textbf{Total Module Mass: 200 kg}
\end{itemize}

\textbf{Process Flow:}

\textbf{Path 1A: Carbon-Composite Filament (Primary Output)}
\begin{enumerate}
    \item \textbf{Grinding:} Load carbon material into sealed ball mill, grind to 10-50 micron powder (2-4 hours)
    \item \textbf{Binder Preparation:} Shred compatible plastic waste, heat to 180-220°C
    \item \textbf{Mixing:} Ratio: 20-40\% carbon powder, 60-80\% polymer binder, blend at 200-250°C (30-60 minutes)
    \item \textbf{Extrusion:} Feed through heated barrel, extrude through 1.75mm or 2.85mm die
    \item \textbf{Spooling:} Wind onto reusable spools, label and store
\end{enumerate}

\textbf{Path 1B: Pure Carbon Products}
\begin{itemize}
    \item Activated Carbon Filters: Compress powder, heat to 800-1000°C
    \item Graphite Components: High-pressure compression, high-temperature treatment (2000-3000°C)
    \item Radiation Shielding Blocks: Mix carbon with regolith simulant
\end{itemize}

\textbf{Energy Profile:}
\begin{itemize}
    \item Power source: Habitat electrical grid
    \item Power consumption: 2-3 kW continuous
    \item Processing time: 4-8 hours per batch (5-10 kg output)
\end{itemize}

\subsection{Module 2: Polymer Recycling Station}

\textbf{Equipment Components:}
\begin{itemize}
    \item Enclosed shredder/grinder with HEPA filtration (80 kg)
    \item Multi-zone heating chamber (60 kg)
    \item Filament extruder with puller (45 kg)
    \item Diameter sensor and feedback control (8 kg)
    \item Vapor capture and carbon scrubber (35 kg)
    \item Spooling mechanism (12 kg)
    \item \textbf{Total Module Mass: 240 kg}
\end{itemize}

\textbf{Process Flow:}
\begin{enumerate}
    \item Material Preparation: Remove contaminants, cut to <10cm pieces
    \item Size Reduction: Grind to 3-8mm pellets/flakes
    \item Drying: Desiccant chamber at 60-80°C (if needed)
    \item Melting: Temperatures by polymer type (160-260°C)
    \item Filtration: Pass through 100-mesh stainless filter
    \item Extrusion: Precision gear pump, temperature-controlled barrel
    \item Cooling \& Diameter Control: Air cooling, servo-controlled puller
    \item Quality Control: Visual inspection, diameter verification
    \item Spooling: Wind onto 1 kg spools, store in sealed containers
\end{enumerate}

\textbf{Safety Systems:}
\begin{itemize}
    \item Vapor Management: Enclosed heating, carbon scrubber filtration
    \item Microplastic Control: Sealed negative-pressure chamber, HEPA filtration
\end{itemize}

\textbf{Energy Profile:}
\begin{itemize}
    \item Power consumption: 1.5-2.5 kW continuous
    \item Processing time: 6-10 hours per 5 kg batch
\end{itemize}

\subsection{Module 3: Metal Processing Station}

\textbf{Equipment Components:}
\begin{itemize}
    \item Heavy-duty metal shredder (100 kg)
    \item Induction furnace with crucible (150 kg)
    \item Wire drawing die set (40 kg)
    \item Casting molds (30 kg)
    \item Cooling system (25 kg)
    \item Metal powder atomizer (optional, 35 kg)
    \item \textbf{Total Module Mass: 380 kg (345 kg without atomizer)}
\end{itemize}

\textbf{Process Flow:}

\textbf{Path 3A: Metal Filament (Wire Drawing Method)}
\begin{enumerate}
    \item Shredding: Cut/shred metal components, separate by type
    \item Melting: Induction crucible (Al: 660°C, Cu: 1085°C, Steel: 1400-1450°C)
    \item Casting: Pour into rod molds (3-5mm diameter)
    \item Wire Drawing: Pull through progressively smaller carbide dies
    \item Quality Control: Diameter measurement, tensile testing
    \item Spooling: Wind onto spools, label
\end{enumerate}

\textbf{Alternative Paths:}
\begin{itemize}
    \item Path 3B: Metal Powder (atomization for advanced printing)
    \item Path 3C: Direct Casting (tools, fasteners, brackets)
\end{itemize}

\textbf{Energy Profile:}
\begin{itemize}
    \item Power consumption: 3-5 kW continuous, 5-7 kW peak
    \item Processing time: 8-12 hours per 3-5 kg batch
\end{itemize}

\subsection{Module 4: Integrated Manufacturing Hub}

\textbf{Equipment Components:}
\begin{itemize}
    \item FDM 3D printer (plastic/composite) (35 kg)
    \item Metal FDM printer with sintering (55 kg)
    \item Direct pellet extruder (40 kg)
    \item CNC finishing station (60 kg)
    \item Quality testing equipment (20 kg)
    \item Parts cleaning/post-processing (15 kg)
    \item \textbf{Total Module Mass: 225 kg}
\end{itemize}

\textbf{Manufacturing Capabilities:}
\begin{itemize}
    \item Plastic/Composite Printing: Uses filament from Modules 1 \& 2
    \item Metal Printing: Uses metal filament from Module 3
    \item Direct Pellet Printing: Uses pellets directly from Module 2
    \item Post-Processing: CNC milling, drilling, surface finishing
\end{itemize}

\textbf{Output Examples:}
\begin{itemize}
    \item Tools \& Equipment: Wrenches, screwdrivers, pliers, utensils
    \item Habitat Components: Storage containers, furniture, ductwork
    \item Life Support Parts: Filter housings, valves, pipe fittings
    \item Scientific Instruments: Sensor housings, sample containers
    \item EVA/Spacesuit Parts: Tool attachments, replacement clips
    \item Structural Elements: Habitat expansion, radiation shielding
\end{itemize}

\section{Supporting Systems}

\subsection{Power Distribution}
\begin{itemize}
    \item Power Source: Habitat electrical grid (nuclear/RTG or other primary power)
    \item Power Requirements:
    \begin{itemize}
        \item Module 1: 2-3 kW continuous
        \item Module 2: 1.5-2.5 kW continuous
        \item Module 3: 3-7 kW (varies with operation)
        \item Module 4: 0.5-2 kW continuous
        \item \textbf{Total system: 7.5-14.5 kW peak demand}
    \end{itemize}
\end{itemize}

\subsection{Environmental Control}
\begin{itemize}
    \item Atmosphere Management: Sealed/vented chambers, vapor capture, HEPA filtration
    \item Temperature Control: Insulation, waste heat recovery, radiator cooling
\end{itemize}

\subsection{Waste Management}
\begin{itemize}
    \item Non-Recyclable Residue: Minimal ($<$2\% of input mass), sealed storage
    \item Filter Maintenance: HEPA (quarterly), activated carbon (regenerated), metal mesh (cleaned)
\end{itemize}

\section{Operations Schedule}

\subsection{Daily Operations (Crew Time: 2-3 hours)}
\begin{itemize}
    \item Morning: Load materials, start grinding/shredding, monitor status
    \item Midday: High-temperature operations, extrusion, quality control
    \item Afternoon: Filament spooling, 3D printing, cleaning
    \item Evening/Night: Long-duration grinds, post-processing, batch planning
\end{itemize}

\subsection{Weekly Maintenance}
\begin{itemize}
    \item Clean all filters and screens
    \item Calibrate diameter sensors
    \item Inspect heating elements
    \item Test emergency shutoffs
    \item Lubricate moving parts
\end{itemize}

\subsection{Monthly Deep Maintenance}
\begin{itemize}
    \item Full system calibration
    \item Replace worn dies/components
    \item Deep clean all chambers
    \item Performance testing and logging
\end{itemize}

\section{Safety Protocols}

\subsection{Hazard Controls}
\begin{itemize}
    \item High Temperature: Thermal barriers, interlocked access, temperature alarms
    \item Moving Machinery: Emergency stops, guards, two-hand operation
    \item Electrical: Ground fault protection, enclosed components, proper labeling
    \item Air Quality: Continuous VOC monitoring, CO detection, particulate sensors
    \item Material Handling: Proper lifting techniques, anti-pinch points, spill containment
\end{itemize}

\subsection{Emergency Procedures}
\begin{itemize}
    \item Fire: CO2 extinguishers, power cutoff, evacuation
    \item Equipment Failure: Immediate shutdown, backup power, automatic cooling
    \item Contamination Release: Seal module, HEPA filtration, crew respirators
\end{itemize}

\section{Performance Metrics}

\subsection{Throughput Capacity}
\begin{itemize}
    \item Module 1 (Carbon): 5-10 kg/day composite filament, 2-5 kg/day pure carbon products
    \item Module 2 (Polymer): 5-8 kg/day plastic filament, $>$90\% Grade A/B quality
    \item Module 3 (Metal): 3-5 kg/day metal wire/rod, 1-2 kg/day metal powder
    \item Module 4 (Manufacturing): 0.5-2 kg/day finished parts, $>$85\% success rate
\end{itemize}

\subsection{Resource Recovery Rates}
\begin{itemize}
    \item Carbon: 95\%+ recovery
    \item Polymers: 85-90\% recovery
    \item Metals: 90-95\% recovery
    \item Fabrics: 70-80\% recovery
    \item Composites: 85-90\% recovery
\end{itemize}

\subsection{Mass Balance Example (Monthly)}
\begin{itemize}
    \item Inputs: 120 kg total (50 kg carbon, 35 kg plastic, 10 kg fabric, 20 kg metal, 5 kg composites)
    \item Outputs: 115 kg usable filament/feedstock/products
    \item Recovery rate: 96\%
\end{itemize}

\section{Crew Training Requirements}

\subsection{Basic Operator (All Crew - 8 hours)}
\begin{itemize}
    \item Safety protocols and PPE
    \item Material sorting and identification
    \item Basic module operation
    \item Emergency shutdown procedures
    \item Quality control basics
\end{itemize}

\subsection{Advanced Operator (Designated Personnel - 40 hours)}
\begin{itemize}
    \item Module-specific technical knowledge
    \item Maintenance procedures
    \item Repair and component replacement
    \item Process optimization
\end{itemize}

\subsection{System Manager (Mission Specialist - 80 hours)}
\begin{itemize}
    \item Full system integration
    \item Complex troubleshooting
    \item Performance analysis and reporting
    \item Modification and upgrades
\end{itemize}

\section{Technology Readiness \& Development Path}

\subsection{Current TRL Levels}
\begin{itemize}
    \item Individual processes: TRL 7-9 (proven on Earth)
    \item Integrated system: TRL 4-5 (lab demonstration)
    \item Mars-specific adaptations: TRL 3-4 (analytical/experimental proof)
\end{itemize}

\subsection{Development Phases}
\begin{enumerate}
    \item Phase 1 (Months 1-6): Earth prototype - build and test individual modules
    \item Phase 2 (Months 7-12): Integration testing - connect all modules
    \item Phase 3 (Months 13-18): Mars analog testing - deploy to Mars Desert Research Station
    \item Phase 4 (Months 19-24): Flight preparation - finalize designs for launch
    \item Phase 5 (Year 3+): Mars deployment - transport, assemble, commission
\end{enumerate}

\section{Advantages for Mars Operations}

\subsection{Resource Independence}
\begin{itemize}
    \item Reduces resupply needs from Earth by 60-80\%
    \item Creates self-sufficiency for consumables
    \item Enables mission extension without additional launches
\end{itemize}

\subsection{CO2 Utilization}
\begin{itemize}
    \item Valuable use for extraction byproduct (carbon)
    \item Closes carbon loop in life support
    \item Reduces carbon waste accumulation
\end{itemize}

\subsection{Adaptability}
\begin{itemize}
    \item Can process unexpected waste streams
    \item Creates custom tools/parts on-demand
    \item Supports mission changes and repairs
\end{itemize}

\subsection{Sustainability}
\begin{itemize}
    \item Zero-waste philosophy
    \item Minimal toxic byproducts
    \item Closed-loop atmosphere management
\end{itemize}

\subsection{Economic Benefits}
\begin{itemize}
    \item Each kg recycled saves $\sim$\$10,000 in launch costs
    \item Monthly savings: \$1,150,000 (115 kg recovered $\times$ \$10k/kg)
    \item System pays for itself in $\sim$2-3 months of operation
\end{itemize}

\section{Scalability \& Future Enhancements}

\subsection{Near-Term Upgrades}
\begin{itemize}
    \item Enhanced Material Library: Ceramics from regolith, glass from silicates, bio-composites
    \item Automation: Robotic sorting, autonomous quality control, self-optimization
    \item Advanced Manufacturing: Multi-material printing, larger build volumes, higher precision
\end{itemize}

\subsection{Long-Term Vision}
\begin{itemize}
    \item Colony Scale (100+ inhabitants): Multiple facilities, specialized production
    \item In-Situ Resource Utilization (ISRU): Process Martian regolith, extract local metals
    \item Distributed Manufacturing: Rover-mounted units, remote site fabrication
\end{itemize}

\section{Compliance Summary}

\begin{itemize}
    \item No incineration/burning - All thermal processing in controlled, non-combustion environment
    \item No toxic emissions - Closed-loop vapor capture with activated carbon filtration
    \item No PFAS generation - System avoids all fluoropolymers and fluorinated compounds
    \item No microplastic release - Sealed grinding with HEPA filtration; all particles captured
    \item No wastewater discharge - Dry processing only; minimal water use with closed-loop drying
    \item Safe for Martian environment - All processes contained; no surface contamination
    \item Crew safety prioritized - Multiple redundant safety systems; comprehensive training
    \item Minimal crew time - 2-3 hours daily operation; automated processes reduce workload
    \item Grid-powered efficiency - Uses habitat electrical supply; no separate power generation needed
    \item Water conservation - Minimal water use in drying processes only; closed-loop system
\end{itemize}

\section{Conclusion}

This integrated recycling-to-manufacturing system transforms Mars habitat waste—including fabrics, food packaging, foam, EVA materials, aluminum structures, and carbon from CO2 extraction—into valuable 3D printing feedstock and finished products. The modular design is powered by the habitat's electrical grid and optimized for Mars conditions while maintaining crew safety and environmental responsibility.

\textbf{Key Innovation:} Leveraging Mars's unique environment (thin atmosphere, low pressure) as advantages rather than obstacles, while processing a comprehensive range of waste materials into useful end products.

\textbf{Mission Impact:} Enables long-duration Mars missions by creating a closed-loop resource economy, reducing Earth dependency, and providing on-demand manufacturing capability for tools, utensils, storage containers, interior habitat outfitting, and habitat expansion.

\textbf{Ready for Development:} All core technologies proven; integration and Mars-specific optimization required before deployment.

\section*{Works Cited}
\begin{itemize}
    \item NASA Mars Facts: \\ \url{https://science.nasa.gov/mars/facts/}
    \item NASA Trash-to-Gas and Trash-to-Supply Gas research: \\ \url{https://ntrs.nasa.gov/api/citations/20130011661/downloads/20130011661.pdf}
    \item MGS-1 Mars Global Simulant \\ \url{https://sciences.ucf.edu/class/simulant_marsglobal/}
\end{itemize}

\end{document}