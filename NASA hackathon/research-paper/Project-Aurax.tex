\documentclass[12pt, a4paper]{article}

% --- Packages ---
\usepackage[utf8]{inputenc}
\usepackage{graphicx}
\usepackage{geometry}
\usepackage{hyperref}
\usepackage{amsmath}
\usepackage{booktabs}
\usepackage{array} % for better column formatting

% --- Page setup ---
\geometry{margin=1in}

\title{\textbf{Mars Habitat Resource Recycling \& Manufacturing System}}
\author{%
    \textbf{Team Name:} Project Aura\textquotesingle x \\[0.3em]
    \textbf{Members:} 
}
\date{\today}

\begin{document}

\maketitle

\section{Problem Definition}

\subsection{Need Statement}
During a three-year mission, an eight-person crew would accumulate over 12,000 kg of inorganic waste, or trash, which is a huge problem. If the issue is not addressed as soon as possible, we humans might cause some permanent damage to the environment on Mars.

\subsection{Goal}
Find an efficient and reliable way to recycle and reuse the waste on Mars.

\subsection{Objectives}

\begin{table}[h!]
\centering
\renewcommand{\arraystretch}{1.3} % line spacing
\setlength{\tabcolsep}{8pt} % column spacing
\begin{tabular}{p{0.25\textwidth} p{0.45\textwidth} p{0.2\textwidth}}
\toprule
\textbf{Objective} & \textbf{Basis for Measurement} & \textbf{Units} \\ 
\midrule
Must be reliable & The endurance of the solution (how many years it can last) & Years \\[0.3em]
Must not be too expensive & The cost to implement the solution & CAD \\[0.3em]
Must minimize the emission & The amount of chemical the solution might emit & -- \\[0.3em]
Must be buildable by no more than eight people & Number of people required for the workload & People \\
\bottomrule
\end{tabular}
\caption{Objectives and Measurement Basis}
\end{table}

\newpage 
\subsection{Constraints}
\begin{itemize}
    \item Solutions must be completed within one week.
    \item Solutions must not burn anything without proper collection on Mars.
    \item Solutions must not cost too much money.
    \item Solutions must not be too complicated to implement.
\end{itemize}

\section{Our Solution}
Mars Habitat Recycling-to-Manufacturing System (MHRMS)
\subsection{Purpose}
This system is designed to convert waste materials from
Mars habitat operations into usable 3D printing filament and
manufacturing feedstock, with emphasis on CO2 extraction 
byproducts. 

\subsection{Design Philofophy}
\begin{itemize}
    \item Modular
    \item relocatable
    \item powered by habitat frid
    \item zero-waste discharge
\end{itemize}

\subsection{Mars Environment Context}
\begin{enumerate}
    \item Mars has a thin atmosphere made up mostly of carbon dioxide, nitrogen, and argon gases, with temperatures ranging from 70°F (20°C) to -225°F (-153°C)
    \item Mars' sparse atmosphere doesn't offer much protection from impacts by meteorites and the thin atmosphere causes heat from the Sun to easily escape
    \item Dust storms can cover much of the planet, sometimes taking months for all dust to settle
    \item Mars completes one rotation every 24.6 hours (a "sol"), with a year lasting 687 Earth days
    \item These conditions directly impact recycling system design: thermal management is critical, dust mitigation essential
\end{enumerate}

\subsection{Design Considerations}
Waste behavior in Mars Environment
\begin{enumerate}
    \item Vacuum Exposure Challenges: Based on NASA Glenn Research Center analysis of waste behavior when exposed to vacuum
    \item Water Sublimation: When waste is exposed to Mars' near-vacuum conditions (0.6\% Earth pressure), approximately 20\% of water content will sublimate before the waste completely freezes. This process takes:
    \begin{itemize}
        \item Small items (3mm droplets): ~7 seconds to freeze
        \item Standard waste "football" (20cm compressed bundle): ~3.4 hours to freeze
    \end{itemize}
    \item Implications for recycling system:
    \begin{enumerate}
        \item Pre-processing must handle both frozen and partially sublimated materials
        \item Vapor capture systems required to prevent water loss during initial processing
        \item Grinding chambers must be sealed to capture sublimating volatiles
        \item Temperature control critical: waste arriving at facility may be frozen solid (-153°C from overnight exposure)
    \end{enumerate}
    \item Thermal management on mars
    \begin{enumerate}
        \item Challenge: Mars temperature swings from 70°F (20°C) to -225°F (-153°C)
        \item Solution:
        \begin{itemize}
            \item Insulated processing chambers: Maintain optimal processing temperatures (150-1400°C depending on module)
            \item Pre-heating systems: Thaw frozen waste before processing
            \item Waste heat recovery: Capture heat from processing for habitat use
            \item Phase-change thermal buffers: Smooth out temperature fluctuations
        \end{itemize}
    \end{enumerate}
    \item Mars sol adaptation
    \begin{enumerate}
        \item Mars day: 24.6 hours VS Earth's 24 hours
        \item Operational impact:
        \begin{itemize}
            \item Processing schedules calibrated to Mars sols
            \item Automation advantage: rovers and processing continue during crew sleep cycles
            \item Benefit: Extra 39 minutes per sol allows for extended processing cycles
        \end{itemize}
    \end{enumerate}
\end{enumerate}

\noindent\textit{Source:} \href{https://science.nasa.gov/mars/facts/}{NASA Mars Facts}.

\section{Input Categories}
\begin{enumerate}
    \item{Category A -- Carbon Materials:}
    \begin{itemize}
        \item Surplus carbon from CO2 extraction (primary feedstock)
        \item Activated carbon filters (spent)
        \item Carbon-containing composites
    \end{itemize}

    \item{Category B -- Polymers/Plastics: }
    \begin{itemize}
        \item Nitrile gloves
        \item Resealable bags (PE/PP)
        \item EVA waste from spacesuits and cargo transfer bags (Nomex, nylon, polyester)
        \item Packaging materials (air cushions, bubble wrap, anti-static bags)
        \item Tape, labels, plastic clips
        \item Foam packaging (Zotek F30, Plastazote foam)
    \end{itemize}

    \item{Category C -- Fabrics: }
    \begin{itemize}
        \item Clothing (cotton, polyester, nylon)
        \item Washcloths and wipes (cellulose, cotton)
        \item Disinfectant wipes 
    \end{itemize}

    \item{Category D -- Food Packaging: }
    \begin{itemize}
        \item Overwrap materials
        \item rehydratable pouches (polyester, polyethylene, aluminum, nylon)
        \item Drink pouches
        \item Thermal pouches
    \end{itemize}

    \item{Category E -- Metals: }
    \begin{itemize}
        \item Filter mesh (stainless steel, aluminum)
        \item Broken tools and fasteners
        \item Structural components from temporary structures
        \item Electronics casings
        \item Wire scraps
        \item Aluminum structures
    \end{itemize}

    \item{Category F -- Composites: }
    \begin{itemize}
        \item Polymer matrix composites with carbon fiber
        \item Multi-layer materials
        \item Fabric-backed materials
        \item Adhesive-bonded assemblies
    \end{itemize}
\end{enumerate}

\section{Alternative Option}
\begin{enumerate}
    \item{Option A -- Direct Venting to Space (NOT RECOMMENDED)} \\
    NASA Glenn Research Center evaluated simply disposing waste via airlock to space. Key findings:
    Technical Challenges:
    \begin{enumerate}
        \item Time-intensive: Using existing airlock technology (3-hour depressurization cycle), disposing of waste from a 180-day, 4-person mission would require 1,250-2,500 hours of crew time
        \item Hazard: 20\% of water content sublimates and can re-condense on spacecraft surfaces or contaminate equipment
        \item Trajectory risk: Low-velocity disposal at L2 libration points risks waste re-impacting spacecraft
    \end{enumerate}
    \textbf{Conclusion:} Direct disposal to space is impractical and hazardous for Mars missions

    \item Option B: Partial Processing to Mixed Gases
    \begin{enumerate}
        \item NASA's "Trash-to-Gas" (TtG) Approach: Process waste in primary reactor only, producing mixed gases:
        \begin{enumerate}
            \item Benefits: Simpler system (fewer processing steps); can provide station-keeping propulsion; lower energy requirements
            \item Limitations: Lower-quality propellant (vs. pure O$_2$/CH$_4$); still requires makeup water; limited manufacturing applications
        \end{enumerate}
    \end{enumerate}
\end{enumerate}

\newpage 
\section*{Works Cited}
\begin{itemize}
    \item cite here
\end{itemize}

\end{document}
